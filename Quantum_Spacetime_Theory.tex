
\documentclass[12pt]{article}
\usepackage{amsmath,amsfonts,amssymb}
\usepackage{graphicx}
\usepackage{hyperref}
\usepackage{authblk}
\usepackage{geometry}
\geometry{margin=1in}

\title{Quantum Spacetime Theory: A Framework for Emergent Geometry and Operator-Level Gravity}
\author{Louis Julius Meric leMerle\\\texttt{vk2icw@gmail.com}}
\date{}

\begin{document}
\maketitle

\begin{abstract}
We propose a new approach to quantum gravity---Quantum Spacetime Theory (QST)---which treats the fabric of spacetime as a discrete, entangled quantum system. Gravity emerges as an operator on a Hilbert space of geometric states. Time is relational, not fundamental. The theory makes experimentally testable predictions, including coherent Planck-scale fluctuations. This paper outlines the theoretical structure, predictions, and proposed experimental tests.
\end{abstract}

\section{Introduction}
Quantum Spacetime Theory (QST) treats geometry itself as quantum information. Each configuration of geometry is a quantum state, and gravity is an operator acting on this state space.

\section{Formal Structure}
Let $\mathcal{H}_{\text{geometry}} = \text{Span} \{ |\Gamma_i\rangle \}$ represent the quantum Hilbert space of geometric states. Gravity acts as:
\[
\hat{G}: \mathcal{H}_{\text{geometry}} \rightarrow \mathcal{H}_{\text{geometry}}
\]

Quantum evolution follows:
\[
P_{i \rightarrow j}(t) = |\langle \Gamma_j | e^{-i \hat{G} t} | \Gamma_i \rangle|^2
\]

Operators may obey:
\[
[\hat{G}_i, \hat{G}_j] = i \hbar \epsilon_{ijk} \hat{G}_k
\]

Entanglement entropy defines geometric boundaries:
\[
S_{\text{ent}}(A) = -\text{Tr}_A(\rho_A \log \rho_A)
\quad \Rightarrow \quad
\langle A_{\text{boundary}} \rangle \propto S_{\text{ent}}(A)
\]

\section{Predictions}
Expected signal structure:
\[
\langle \delta x^2 \rangle \sim L_p^2 \left( \frac{L}{L_p} \right)^{1 - \alpha}
\]

\section{Experimental Proposal}
QST predicts measurable Planck-scale fluctuations. Ground-based QSI and space-based QST-SI systems are proposed.

\section{References}
\begin{itemize}
\item Green, Schwarz, Witten. \textit{Superstring Theory}. Cambridge (1987).
\item Rovelli. \textit{Quantum Gravity}. Cambridge (2004).
\item Verlinde. ``On the Origin of Gravity''. JHEP 1104:029 (2011).
\item Hogan. ``Indeterminacy of Holographic Quantum Geometry''. PRD 78 (2008).
\item Fermilab Holometer. Class. Quant. Grav. 36 (2019).
\end{itemize}

\end{document}
